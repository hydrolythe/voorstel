%==============================================================================
% Sjabloon onderzoeksvoorstel bachelorproef
%==============================================================================
% Gebaseerd op LaTeX-sjabloon ‘Stylish Article’ (zie voorstel.cls)
% Auteur: Jens Buysse, Bert Van Vreckem
%
% Compileren in TeXstudio:
%
% - Zorg dat Biber de bibliografie compileert (en niet Biblatex)
%   Options > Configure > Build > Default Bibliography Tool: "txs:///biber"
% - F5 om te compileren en het resultaat te bekijken.
% - Als de bibliografie niet zichtbaar is, probeer dan F5 - F8 - F5
%   Met F8 compileer je de bibliografie apart.
%
% Als je JabRef gebruikt voor het bijhouden van de bibliografie, zorg dan
% dat je in ``biblatex''-modus opslaat: File > Switch to BibLaTeX mode.

\documentclass{voorstel}

\usepackage{lipsum}

%------------------------------------------------------------------------------
% Metadata over het voorstel
%------------------------------------------------------------------------------

%---------- Titel & auteur ----------------------------------------------------

% TODO: geef werktitel van je eigen voorstel op
\PaperTitle{De aanwending van het structuralisme op muziekvoorkeursalgoritmen}
\PaperType{Onderzoeksvoorstel Bachelorproef 2019-2020} % Type document

% TODO: vul je eigen naam in als auteur, geef ook je emailadres mee!
\Authors{Guylian Bollon\textsuperscript{1}} % Authors
\CoPromotor{???\textsuperscript{2} (Bedrijfsnaam)}
\affiliation{\textbf{Contact:}
  \textsuperscript{1} \href{mailto:guylian.bollon.y2421@student.hogent.be}{guylian.bollon.y2421@student.hogent.be};
}

%---------- Abstract ----------------------------------------------------------

\Abstract{In mijn bachelorproef zal ik proberen de theoretische fundamenten van een algoritme te ontwikkelen op basis van het structuralisme. Dit algoritme zal dan worden toegepast om muziekplaylists te genereren. De opportuniteiten die kunnen benut worden bij het in gebruik van dit algoritme t.o.v. andere algoritmen die gebruikt worden voor dezelfde taak worden hierin ook weergegeven. Daarmee zal het nut van dit algoritme voor het doen van deze taak ook 
}

%---------- Onderzoeksdomein en sleutelwoorden --------------------------------
% TODO: Sleutelwoorden:
%
% Het eerste sleutelwoord beschrijft het onderzoeksdomein. Je kan kiezen uit
% deze lijst:
%
% - Mobiele applicatieontwikkeling
% - Webapplicatieontwikkeling
% - Applicatieontwikkeling (andere)
% - Systeembeheer
% - Netwerkbeheer
% - Mainframe
% - E-business
% - Databanken en big data
% - Machineleertechnieken en kunstmatige intelligentie
% - Andere (specifieer)
%
% De andere sleutelwoorden zijn vrij te kiezen

\Keywords{Onderzoeksdomein. Machineleertechnieken en kunstmatige intelligentie --- Structuralisme --- Muziekplaylists} % Keywords
\newcommand{\keywordname}{Sleutelwoorden} % Defines the keywords heading name

%---------- Titel, inhoud -----------------------------------------------------

\begin{document}

\flushbottom % Makes all text pages the same height
\maketitle % Print the title and abstract box
\tableofcontents % Print the contents section
\thispagestyle{empty} % Removes page numbering from the first page

%------------------------------------------------------------------------------
% Hoofdtekst
%------------------------------------------------------------------------------

% De hoofdtekst van het voorstel zit in een apart bestand, zodat het makkelijk
% kan opgenomen worden in de bijlagen van de bachelorproef zelf.
%---------- Inleiding ---------------------------------------------------------

\section{Introductie} % The \section*{} command stops section numbering
\label{sec:introductie}

Het structuralisme als gedachtengoed is er een die in een diversiteit van disciplines aanwezig is. Van linguïstiek tot antropologie en literaire kritiek. Echter heeft men zich nooit de vraag gesteld hoe deze aan te wenden binnenin computerwetenschappen. Men zou misschien menen dat omdat het structuralisme enkel aangewend is binnenin de sociale wetenschappen en nooit geformaliseerd is geweest is tot een systeem binnenin het instituut van de computerwetenschappen dat het van geen nut is om te leren om deze aan te wenden binnenin een andere institutionele context. Echter is dit een te snel getrokken conclusie. Het structuralisme bestaat uit een verzameling positieve claims die juist binnenin de computerwetenschappen geformaliseerd kunnen worden tot een wiskundig model dat kan gecontrasteerd worden met andere al bestaande algoritmes en kan pijlen naar hun efficiëntie. Iets beters hebben dan het al bestaande zou al een grote stap vooruit zijn en al zou het algoritme minder goed werken dan al bestaande algoritmen zou het nog steeds de moeite waard zijn om dat aangekaart te zien. 

De onderzoeksvraag zal dus vanuit deze optiek de deze zijn: Hoe vaardig is een algoritme dat zich baseert op structuralisme om een playlist te genereren dat de voorkeuren van de gebruiker weergeeft? 

%---------- Stand van zaken ---------------------------------------------------

\section{State-of-the-art}
\label{sec:state-of-the-art}

Een van de meest recente experimenten omtrent het gebruik van algoritmes om muziekplaylists te generen is van Akimchuk et al. (2019) Hierbij werden er 3 soorten van algoritmen besproken. Bij de eerste probeert men de voorkeuren van de personen voor een bepaald muziekstuk te bepalen op basis van hoe gelijkaardig de muziek is t.o.v. andere muziekstukken (IIN). Bij de tweede soort probeert men de muziekvoorkeuren van een persoon te genereren door de foutmarge te minimaliseren die voorkomt wanneer een persoon een bepaald muziekstuk niet leuk vindt (WMRF). De derde manier is door op basis van bayesiaans te redeneren te voorspellen wat de kans is dat iemand een ander muziekstuk leuk zal vinden op basis van zijn voorkeuren aangegeven door voormalige muziekstukken (BPR). In dit onderzoek probeerde men te zien hoe efficiënt de algoritmes erin slaagden om muziek te produceren die de mensen leuk vonden. Verder werden er nog als vergelijkingspunten een algoritme genomen dat alle muziekstukken willekeurig sorteert en een dat alle muziekstukken sorteert naar hun populariteit. Het resultaat is dat IIN de beste manier is om muziekstukken te sorteren. 

% Voor literatuurverwijzingen zijn er twee belangrijke commando's:
% \autocite{KEY} => (Auteur, jaartal) Gebruik dit als de naam van de auteur
%   geen onderdeel is van de zin.
% \textcite{KEY} => Auteur (jaartal)  Gebruik dit als de auteursnaam wel een
%   functie heeft in de zin (bv. ``Uit onderzoek door Doll & Hill (1954) bleek
%   ...'')

%---------- Methodologie ------------------------------------------------------
\section{Methodologie}
\label{sec:methodologie}

Er zal een groep van mensen genomen worden die in 3 subgroepen onderverdeeld zullen worden. De eerste zal zijn muziekstukken gesorteerd zien volgens een structuralistisch algoritme. De tweede zal ze gesorteerd zien volgens een IIN algoritme. De derde zal ze willekeurig gesorteerd zien. De proefpersonen zullen elk gedurende 7 dagen 1 uur naar de muziek luisteren, waarbij de eerste groep na iedere keer dat ze hebben beluisterd een woord achterlaat dat zijn idee over het muziekstuk weergeeft (zo'n iemand zal ook de optie krijgen om een al ingegeven woord opnieuw in te geven), en alle groepen erna het lied beoordelen. Op het einde zullen ze dan in een vragenlijst invullen hoe plezant hun ervaring in het algemeen was. 

De reden waarom we voor de eerste groep kiezen om nog een woord te gebruiken om hun mening over de muziek te formuleren is omdat een structuralistisch algoritme enkel kan beginnen verbanden te leggen als het muziekstuk in contrast tot andere muziekstukken gedefinieerd is. Dit omdat de relaties arbitrair zijn en vooral t.o.v. elkaar gedefinieerd binnenin het structuralisme \autocite(Saussure). 

%---------- Verwachte resultaten ----------------------------------------------
\section{Verwachte resultaten}
\label{sec:verwachte_resultaten}

Ik verwacht dat het verschil in voorkeur tussen de 3 systemen statistich belangrijk is en dat mensen de voorkeur geven aan het tweede systeem boven het eerste systeem en aan het eerste systeem boven het derde systeem. 

%---------- Verwachte conclusies ----------------------------------------------
\section{Verwachte conclusies}
\label{sec:verwachte_conclusies}

Playlists structuralistisch genereren is beter dan een playlist wiens imput gemaakt werd door random items, maar is een slechtere keuze dan al bestaande algoritmes. 

%------------------------------------------------------------------------------
% Referentielijst
%------------------------------------------------------------------------------
% TODO: de gerefereerde werken moeten in BibTeX-bestand ``voorstel.bib''
% voorkomen. Gebruik JabRef om je bibliografie bij te houden en vergeet niet
% om compatibiliteit met Biber/BibLaTeX aan te zetten (File > Switch to
% BibLaTeX mode)

\phantomsection
\printbibliography[heading=bibintoc]

\end{document}
